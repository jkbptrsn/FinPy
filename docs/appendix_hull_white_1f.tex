\section{One-factor Hull-White Model}
The one-factor Hull-White model \cite{Hull1990} is an extension of the Vasicek model such that the parameters are allowed to be time-dependent, i.e.
\begin{equation}
dr_t = \kappa_t \left( \theta_t - r_t \right) dt + \sigma_t dW_t.
\end{equation}
where the SDE parameters have the same interpretation as in the Vasicek model, and, again, $W_t$ denotes a one-dimensional Wiener process under the risk-neutral $Q$-measure.

For the Hull-White model to be consistent with the initial yield curve, $\theta_t$ is given by \cite{Andersen2010}
\begin{equation}
\theta_t = \frac{1}{\kappa_t} \frac{\partial f(0,t)}{\partial t} + f(0,t) + \frac{1}{\kappa_t} \int_0^t \e^{-2\int_u^t \kappa_v dv} \sigma_u^2 du.
\end{equation}

Now we define a pseudo short rate as $x_t = r_t - f(0,t)$ such that
\begin{equation}
dx_t = \left( y_t - \kappa_t x_t \right) dt + \sigma_t dW_t, \;\;\; x_0 = 0,
\label{eq:HullWhiteSDEx}
\end{equation}
where
\begin{equation}
y_t = \int_0^t \e^{-2\int_u^t \kappa_v dv} \sigma_u^2 du,
\end{equation}
thus, the initial yield curve is used explicitly in the simulation of the pseudo 
short rate, and any issues with yield curve interpolation are avoided.

The SDE in Eq.~(\ref{eq:HullWhiteSDEx}) can be solved analytically by introducing the function
\begin{equation}
h(t, x_t) = x_t \e^{\int_0^t \kappa_u du},
\end{equation}
using It\^{o}'s lemma
\begin{eqnarray}
dh(t, x_t) &=& \frac{\partial h}{\partial t} dt + \frac{\partial h}{\partial x_t} dx_t + \frac{1}{2}\frac{\partial^2 h}{\partial x_t^2} \left(dx_t\right)^2 \\
&=& \kappa_t h(t, x_t) dt + \e^{\int_0^t \kappa_u du} dx_t \\
&=& y_t \e^{\int_0^t \kappa_u du} dt + \sigma_t \e^{\int_0^t \kappa_u du} dW_t,
\end{eqnarray}
and integrating from $t_1$ to $t_2$
\begin{equation}
h(t_2, r_{t_2}) = h(t_1, r_{t_1}) + \int_{t_1}^{t_2} y_t \e^{\int_0^t \kappa_u du} dt + \int_{t_1}^{t_2} \sigma_t \e^{\int_0^t \kappa_u du} dW_t,
\end{equation}
such that
\begin{equation}
x_{t_2} = x_{t_1} \e^{-\int_{t_1}^{t_2} \kappa_u du}  + \int_{t_1}^{t_2} y_t \e^{-\int_t^{t_2} \kappa_u du} dt + \int_{t_1}^{t_2} \sigma_t \e^{-\int_t^{t_2} \kappa_u du} dW_t.
\end{equation}

As for the Vasicek model, the pseudo short rate process is Gaussian with conditional mean
\begin{equation}
\E_{t_1}\! \left[ x_{t_2} \right] = x_{t_1} \e^{-\int_{t_1}^{t_2} \kappa_u du}  + \int_{t_1}^{t_2} y_t \e^{-\int_t^{t_2} \kappa_u du} dt,
\end{equation}
and, using the It\^{o} isometry, the conditional variance of the process becomes
\begin{eqnarray}
\Var_{t_1}\! \left[ x_{t_2} \right] &=& \E_{t_1}\! \left[ \left( \int_{t_1}^{t_2} \sigma_t \e^{-\int_t^{t_2} \kappa_u du} dW_t \right)^2 \right] \\
&=& \int_{t_1}^{t_2} \sigma_t^2 \e^{-2\int_t^{t_2} \kappa_u du} dt.
\end{eqnarray}

The corresponding pseudo discount factor is written as
\begin{equation}
D_{t} = \e^{I_t},
\end{equation}
where
\begin{eqnarray}
I_t &=& -\int_0^t x_u du \\
&=& - x_{0} \int_0^t \e^{-\int_{0}^{u} \kappa_v dv} du - \int_0^t\int_{0}^{u} y_v \e^{-\int_v^{u} \kappa_s ds} dv du \\
&& - \int_0^t \int_{v}^{t} \sigma_v \e^{-\int_v^{u} \kappa_s ds} du dW_v,
\end{eqnarray}
where the order of integration has been changed in the last term.

The discount factor process is likewise Gaussian with conditional mean
\begin{equation}
\E_{t_1}\! \left[ I_{t_{2}} \right] = I_{t_1} - x_{t_1} \int_{t_1}^{t_2} \e^{-\int_{t_1}^{u} \kappa_v dv} du - \int_{t_1}^{t_2} \int_{t_1}^{u} y_v \e^{-\int_v^{u} \kappa_s ds} dv du,
\end{equation}
and, using the It\^{o} isometry, the conditional variance becomes
\begin{eqnarray}
\Var_{t_1}\! \left[ I_{t_{2}} \right] &=& \E_{t_1}\! \left[ \left( \int_{t_1}^{t_2} \int_{v}^{t_2} \sigma_v \e^{-\int_v^{u} \kappa_s ds} du dW_v \right)^2 \right] \\
&=& \int_{t_1}^{t_2} \sigma_v^2 \left( \int_{v}^{t_2} \e^{-\int_v^{u} \kappa_s ds} du \right)^2 dv.
\end{eqnarray}

\textcolor{red}{Compare (A.50) with Andersen \& Piterbarg!}

Again, we can write both $x_t$ and $I_t$ as a sum of a deterministic term ($D$) and a stochastic term ($S$)
\begin{eqnarray}
x_t &=& D_{x_t} + S_{x_t} \\
I_t &=& D_{I_t} + S_{I_t},
\end{eqnarray}
such that the covariance of $x_t$ and $I_t$ becomes
\begin{eqnarray}
\Cov_{t_1}\! \left[ x_{t_2}, I_{t_2} \right] &=& \E_{t_1}\! \left[ S_{x_{t_2}} S_{I_{t_2}} \right] \\
&=& - \int_{t_1}^{t_2} \int_{t_1}^u \sigma_s^2 \e^{-\int_s^{u} \kappa_v dv} \e^{-\int_s^{t_2} \kappa_v dv} ds du.
\end{eqnarray}

\textcolor{red}{Derive (A.54), and compare with Andersen \& Piterbarg!}

\subsection{Transformation from pseudo representation}

The initial discount curve and instantaneous forward rate curve are related by
\begin{equation}
P(0,t) = \e^{- \int_0^t f(0,u) du}.
\end{equation}

Using the transformation of the short rate $r_t = x_t + f(0,t)$, we get that
\begin{eqnarray}
P(0,t) &=& \E \left[ \e^{- \int_0^t r_u du} \right] \\
&=& \e^{- \int_0^t f(0,u) du} \E \left[ \e^{- \int_0^t x_u du} \right],
\end{eqnarray}
which implies that
\begin{equation}
\E \left[ \e^{- \int_0^t x_u du} \right] = 1.
\end{equation}

For a given scenario, i.e., $r_t(\omega) = x_t (\omega) + f(0,t)$, we get the 
time $t$ stochastic discount factor as
\begin{equation}
\e^{- \int_0^t r_u (\omega) du} = P(0,t) \e^{- \int_0^t x_u (\omega) du}.
\end{equation}